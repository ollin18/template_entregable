
%%%%%%%%%%%%%%%%%%%%%%%%%%%%%%%%%%%%%%%%%%%%%%%%%%%%%%%%%%%%%%%%%%%%%%%%%%%%%%%%%%
% First Page
%%%%%%%%%%%%%%%%%%%%%%%%%%%%%%%%%%%%%%%%%%%%%%%%%%%%%%%%%%%%%%%%%%%%%%%%%%%%%%%%%%
\graphicspath{{images/}}

\newcommand{\MyName}{Ollin Demian Langle Chimal}
\newcommand{\Institution}{Laboratorio de Datos, SEDESOL}
\newcommand{\DelTitle}{Modelo}
\newcommand{\DelNumber}{2}
\newcommand{\DelVersion}{0.1/1.0}
\newcommand{\Contrato}{ATN/OC 15822-RG}
\newcommand{\footertext}{\raisebox{3mm}{Entregable \DelNumber}}
\setlength{\footheight}{36pt}
\newcommand{\footerlogo}{\raisebox{3mm}{\leavevmode\includegraphics[width=2cm]{bid}}}
\clearscrheadfoot
\pagestyle{empty}


% \begin{tikzpicture}[overlay,remember picture]
%   \draw [line width=1pt]
%   ($ (current page.north west) + (1cm,-1cm) $)
%   rectangle
%   ($ (current page.south east) + (-1cm,1cm) $);
% \end{tikzpicture}

\definecolor{SINetblue}{HTML}{07505B}
\newcolumntype{C}{ >{\centering\arraybackslash} m{4cm} }

\begin{center}
SEDESOL 2018
\vspace{0.1cm}

  \begin{center}


  % H2020 has no logo and no visual identity

  % \includegraphics[width=0.7\textwidth]{images/LOGO_FJR}
      \Large \MyName \\\vspace{5mm}
\begin{multicols}{2}
\includegraphics[width=0.35\textwidth]{images/bid}
\includegraphics[width=0.25\textwidth]{images/LOGO_FJR}
\end{multicols}
\begin{multicols}{2}
\includegraphics[width=0.4\textwidth]{images/sedesol}
\includegraphics[width=0.2\textwidth]{images/presidencia}
\end{multicols}


  \vspace{2mm}

  \end{center}
  \vspace{0.3cm}
  {\Large Título del proyecto: USO DE DATOS MASIVOS PARA LA EFICIENCIA DEL ESTADO Y LA INTEGRACIÓN REGIONAL\\}
    {\large Clave: \Contrato}\\
  \vspace{0.5cm}
  \Large Puesto: Científico de Datos Senior\\
  \vspace{1.0cm}

  \begin{spacing}{2.5}
    \textbf{\Huge \DelTitle}\\\vspace{10mm}
    \textbf{\Large Entregable número: \DelNumber} \\\vspace{10mm}
  \end{spacing}

  % \vspace*{\fill}

  %just to avoid warning :)
  \newcommand\undefcolumntype[1]{\expandafter\let\csname NC@find@#1\endcsname\relax}
  \newcommand\forcenewcolumntype[1]{\undefcolumntype{#1}\newcolumntype{#1}}
  \forcenewcolumntype{C}{ >{\arraybackslash} m{3cm} }


  % \begin{tabular}{C@{\hspace*{0cm}}l}
  %   % \includegraphics[scale=0.2]{images/logos/EU_Flag_320_213} &
  %   % \begin{tabular}{l}
  %   % {Funded by the European Union’s Horizon 2020 research and innovation programme}\\
  %   % {under the Marie Sklodowska-Curie Grant Agreement No. 699924}\\
  %   \end{tabular}
  % \end{tabular}
\end{center}

\clearpage

%%%%%%%%%%%%%%%%%%%%%%%%%%%%%%%%%%%%%%%%%%%%%%%%%%%%%%%%%%%%%%%%%%%%%%%%%%%%%%%%%%
% Second Page
%%%%%%%%%%%%%%%%%%%%%%%%%%%%%%%%%%%%%%%%%%%%%%%%%%%%%%%%%%%%%%%%%%%%%%%%%%%%%%%%%%
\setlength{\headheight}{0.7cm}
\setlength{\footskip}{18mm}
\addtolength{\textheight}{-\footskip}
\pagestyle{empty}

\begin{flushright}
    \begin{tabular}{lp{11cm}}
    \textbf{Acrónimo del proyecto:}       &   Estimación de Ingreso \\
    \hline
    \textbf{Nombre completo del proyecto:} & USO DE DATOS MASIVOS PARA LA EFICIENCIA DEL ESTADO Y LA INTEGRACIÓN REGIONAL\\
    \hline
    \textbf{Referencia:}                  &   ATN/OC 15822-RG\\
    \hline
%    \textbf{Topic:}                 &   ICT-10-2015 \\
%    \textbf{Type of Action:}        &   RIA \\
    % \textbf{Grant Number:}          &   699924 \\
    \textbf{URL del Proyecto:}           &  \url{http://www.plataformapreventiva.gob.mx}
  \end{tabular}



% define "struts", as suggested by Claudio Beccari in
%    a piece in TeX and TUG News, Vol. 2, 1993.
\newcommand\Tstrut{\rule{0pt}{2.6ex}}         % = `top' strut
\newcommand\Bstrut{\rule[-0.9ex]{0pt}{0pt}}   % = `bottom' strut

  \begin{tabular}{|l|p{115mm}|}\hline
    % \Tstrut\Bstrut Editor:& XXX, XXX-Institution\\\hline
    \Tstrut\Bstrut Tipo de Entregable:& Reporte (R) \\\hline
    % \Tstrut\Bstrut Dissemination level:& Public (PU)\\\hline
    \Tstrut\Bstrut Fecha de Entrega Contractual:& Agosto - 2018\\\hline
    \Tstrut\Bstrut Fecha de Entrega& Agosto - 2018\\\hline
    % \Tstrut\Bstrut Suggested Readers:&Project partners, future community-lab.net users\\\hline
    \Tstrut\Bstrut Número de Páginas:&\pageref{finalpg}\\\hline
    \Tstrut\Bstrut Keywords:& estimación ingreso ciencia datos R paquete anomalias \\\hline
    \Tstrut Autor:&
    \begin{tabular}[t]{l}
      \MyName, \Institution \Bstrut \\
      % XXX - YYY, Institution \\
      % XXX - YYY, Institution \\
      % XXX - YYY, Institution \Bstrut \\
    \end{tabular}\\\hline
    % \Tstrut\Bstrut Peer review: &
    % \begin{tabular}[t]{l}
    %             ZZZ - Institution\\
    %     \end{tabular}\\\hline
  \end{tabular}
\end{flushright}

\section*{Resumen}
El uso de nuevas tecnologías es fundamental para la incorporación del análisis de datos masivos como auxiliar en la toma de decisiones de la política pública. La extensa gama de servicios existentes genera una gran dificultad de dominio general de las mismas por lo que resulta imperativo un esfuerzo para conjuntarlas de manera sencilla   en una misma herramienta que utilice el mismo lenguaje unificado. De esta manera se construyó un paquete del lenguaje de programación \textit{R} con dichas características.\\
Las irregularidades en los padrones de beneficiarios son usualmente difíciles de detectar pues la información en general no se encuentra establecida de una forma ordenada lo cual hace que no se pueda estudiar de forma sistemática para todos los padrones, entidades, programas o demás claves de agregación. La cantidad de información que se debe procesar para generar las estadísticas necesarias también agrega un nivel dificultad a esta tarea pues el hecho de trabajar directamente con sets de datos que en su conjunto suman ccerca de un Terabyte de información por lo que hay que idear estrategias para que estos se puedan analizar a un costo accesible también agrega un nivel dificultad  a esta tarea pues el hecho de trabajar directamente con sets de datos que en su conjunto suman ccerca de un Terabyte de información por lo que hay que idear estrategias para que estos se puedan analizar a un costo accesible.
\clearpage

