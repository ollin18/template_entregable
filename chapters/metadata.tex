\chapter{Metadatos}
\subsection{Metadatos}
Los metadatos sirven para entender los datos, lo cual es tan importante como los datos mismos por lo que es importante contar con diccionarios y catálogos que den información sobre la naturaleza de ellos.
\subsubsection{Diccionarios}
Los diccionarios tienen la siguiente estructura

\begin{longtable}{|p{8cm}|p{8cm}|}
    \hline
    \hline
    Campo  & Descripción\\
    \hline
    \multicolumn{2}{l}{Identificadores únicos}\\
    \hline
    ID & Nombre de la columna \\
    \hline
    Nombre & Lo que representa el ID \\
    \hline
    Fuente & Fuente original de los datos \\
    \hline
    URL & Dirección url de la fuente si es que existe \\
    \hline
    Tipo & Tipo de la fuente de datos \\
    \hline
    Subtipo & Mayor granularidad del tipo de la fuente \\
    \hline
    Periodo & Periodo de descarga \\
    \hline
    Actualización\_sedesol & La fecha en que se hizo una actualización de la base de datos\\
    \hline
    Metadata & Explicación de la variable \\
    \hline
    data\_date & Fecha intrínseca al source script de descarga \\
    \hline
\end{longtable}
\caption{Diccionario de datos}
\label{tab:diccionario}

\subsubsection{Catálogo}
Un catálogo de datos o DCAT es un documento con la información necesaria para entender el origen y la naturaleza de los datos en cuestión. Consta de un formato predefinido de manera homologada.
