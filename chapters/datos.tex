\chapter{Fuentes de datos}
El objetivo del proyecto ATN/OC 15822-RG en general es el realizar una mejor focalización de los recursos para aquellos hogares que en verdad se encuentren en estado de carencia para así poder hacer una mayor diferencia, un objetivo particular también es el de realizar una estimación de ingreso con las variables con las que se cuentan, las cuales se obtienen a partir de las fuentes que se mencionaron en el capítulo de introducción.\\
\section{SIFODE}
SIFODE cuenta con una recopilación de datos de CUIS tomando en cuenta ciertas variables que son de interés para distintos ámbitos del propio sistema, el total de personas que están registradas en el mismo asciende a 39.9 millones.
\subsection{SIFODE Calificación}
Se le da el nombre de calificación a un set de datos que se utiliza para definir universos potenciales de participantes del SIFODE a partir de unas variables socioeconómicas prestablecidas. Como se ha mencionado anteriormente, la unidad básica es el hogar, por lo que existe un identificador único para cada uno de ellos que es llamado \textit{LLAVE\_HOGAR\_H}, dicho identificador consta de 50 caracteres que identifican encuestas únicas y está integrado por el o los folios que identifican a una encuesta de manera única declarados por programa, a partir de este se genera otro identificador para cada una de las personas que componen el hogar, llamado \textit{C\_INTEGRANTE}, el cual consta de números consecutivos del 1 al número total de integrantes del hogar. Por último, existe una variable llamada \textit{NEW\_ID} el cual es un identificador único de personas en padrones correspondiente a SIFODE, este se construye a partir del nombre, apellido paterno, apellido materno, fecha de nacimiento, entidad de nacimiento, sexo y CURP.\\
Las bases históricas de CUIS tienen \textit{LLAVE\_HOGAR\_H} y \textit{C\_INTEGRANTE} y las bases de SIFODE tienen \textit{LLAVE\_HOGAR\_H}, \textit{C\_INTEGRANTE} e \textit{ID\_UNICO\_39\_9}, que es el mismo que el \textit{NEW\_ID} por lo que es de esta manera en que se pueden relacionar CUIS, SIFODE y PUB.

\begin{longtable}{|p{8cm}|p{8cm}|}
    \hline
    \hline
    Campo  & Descripción\\
    \hline
    \multicolumn{2}{l}{Identificadores únicos}\\
    \hline
    LLAVE\_HOGAR\_H & Identifica encuestas únicas integradas por el o los folios que identifican a una encuesta de manera única declaradas por el programa\\
    \hline
    C\_INTEGRANTE & Número de renglón de integrante en la encuesta\\
    \hline
    ID\_UNICO\_39\_9 & Identificador Único de Personas en Padrones corresponiente a SIFODE 39.9\\
    \hline
    \hline
    \multicolumn{2}{l}{Indentificación de personas}\\
    \hline
    NUME\_PER & Informante adecuado \\
    \hline
    TIE\_CURP & La persona tiene CURP \\
    \hline
    NB\_CURP & CURP \\
    \hline
    NB\_NOMBRE &  Nombre \\
    \hline
    NB\_PRIMER\_AP & Primer apellido \\
    \hline
    NB\_SEGUNDO\_AP & Segundo apellido \\
    \hline
    FCH\_NACIMIENTO & Fecha de nacimiento \\
    \hline
    C\_CD\_EDO\_NAC & Entidad de nacimiento \\
    \hline
    C\_CD\_SEXO & Sexo \\
    \hline
    EDAD & Años cumplidos al momento de la encuesta \\
    \hline
    EDAD\_ACTUAL & Edad actual calculada a 2017 \\
    \hline
    VAL\_NB\_RENAPO & Validación del CURPO con RENAPO \\
    \hline
    ANIO\_NAC & Año de nacimiento \\
    \hline
    \hline
    \multicolumn{2}{l}{Clasificación de pobreza y carencias}\\
    \hline
    POBREZAP & Personas en pobreza \\
    \hline
    POB\_EXTREMP & Personas en pobreza extrema \\
    \hline
    POB\_EXTREM\_ALIMP & Personas en pobreza extrema alimentaria \\
    \hline
    POB\_MODP & Personas en pobreza moderada \\
    \hline
    VUL\_CARENP & Personas vulerables por carencia \\
    \hline
    VUL\_INGRESOP & Personas vulnerables por ingresos \\
    \hline
    NP\_NVP & Personas no pobres y no vulnerables \\
    \hline
    CARENCIADAP & Personas con al menos una carencia \\
    \hline
    CARENCIAS3P & Personas con al menos tres carencias \\
    \hline
    IC\_REZEDU\_1 & Indicador de carencia por rezago educativo \\
    \hline
    IC\_REZEDU15M & Personas de 3 a 15 años con rezago educativo \\
    \hline
    IC\_REZEDU\_81 & Personas de 16 años o más que nacieron antes de 1982 y que no tienen primaria completa \\
    \hline
    IC\_REZEDU\_82 & Personas de 16 años o más que nacieron antes de 1982 y que no tienen secundaria completa \\
    \hline
    JUBILADO\_IC & Cuenta con jubilación o pensión como prestación laboral \\
    \hline
    IC\_ASALUD & Indicador de acceso a servicios de salud (SIFODE)\\
    \hline
    SEG\_POP\_1 & Personas afiliadas al Seguro Popular \\
    \hline
    IMSS\_1 & Personas afiliadas al IMSS \\
    \hline
    ISSSTE\_1 & Personas afiliadas al ISSSTE \\
    \hline
    PEMEX\_1 & Personas afiliadas al PEMEX, Marina o Defensa \\
    \hline
    PRIVA\_1 & Personas afiliadas a una clínica u hospital privado \\
    \hline
    SERV\_SAL\_99 & Personas que no cuentan con servicio de salud \\
    \hline
    SERV\_SAL & Acceso a servicio de salud \\
    \hline
    IC\_SS & Indicador de carencia por acceso a la seguridad social \\
    \hline
    SSPEI & Personas económicamente inactivas sin acceso a seguridad social \\
    \hline
    SSPEA & Personas económicamente activas sin acceso a seguridad social \\
    \hline
    SSINTOCUP & Personas ocupadas sin acceso a seguridad social \\
    \hline
    SSAM & Adultos mayores sin acceso a seguridad social\\
    \hline
    IC\_CV & Indicador de carencia por calidad y espacios de la vivienda \\
    \hline
    IC\_PISO & Indicador de carencia del material de piso \\
    \hline
    IC\_TECH & Indicador de carencia del material de techo \\
    \hline
    IC\_MURO & Indicador de carencia del material de muros \\
    \hline
    IC\_HAC & Indicador de carencia por hacinamiento \\
    \hline
    IC\_SBV & Indicador de carencia en el acceso a los servicios básicos de la vivienda \\
    \hline
    ISBV\_AGUA & Indicador de carencia de acceso al agua \\
    \hline
    ISBV\_DREN & Indicador de carencia de servicio de drenaje \\
    \hline
    ISBV\_LUZ & Indicador de carencia de servicio de electricidad \\
    \hline
    ISBV\_COMBUS & Indicador de carencia de servicio de combustible para cocinar \\
    \hline
    IC\_ALI & Indicador de carencia por acceso a la alimentación \\
    \hline
    INSEG\_ALIM & Personas que cuentan con inseguridad alimentaria \\
    \hline
    IA\_AD & Inseguridad alimentaria en adultos \\
    \hline
    IA\_MEN & Inseguridad alimentaria en menores \\
    \hline
    SUMCARENP & Número de carencias a nivel persona \\
    \hline
    POB\_LBM & Personas con ingreso estimado inferior a la LBM \\
    \hline
    POB\_LB & Personas con ingreso estimado inferior a la LB \\
    \hline
    \hline
    \multicolumn{2}{l}{Información socioeconómica de personas}\\
    \hline
    C\_LENGUA\_IND & Lengua indígena \\
    \hline
    OTRO\_DIAL & Especificar otra lengua indígena\\
    \hline
    HABL\_ESP & También habla español \\
    \hline
    INDIGENA & Se considera indígena \\
    \hline
    LEER\_ESCR & Sabe leer escribir \\
    \hline
    C\_ULT\_NIVEL & Nivel escolar (último aprobado) \\
    \hline
    C\_ULT\_GRA & Grado (años aprobados) \\
    \hline
    NIV\_ED & Nivel educativo \\
    \hline
    ASIS\_ESC & Actualmente asiste a la escuela \\
    \hline
    C\_CON\_TRA & Condición laboral \\
    \hline
    C\_POS\_OCUP & Ocupación principal \\
    \hline
    C\_CON\_RES & Condición de residencia \\
    \hline
    C\_CD\_PARENTESCO & Parentesco al jefe(a) del hogar \\
    \hline
    PADRE & Renglón de identificación del padre en el hogar \\
    \hline
    MADRE & Renglón de identificación de la madre en el hogar \\
    \hline
    C\_CD\_EDO\_CIVIL & Estado civil \\
    \hline
    CONYUGE & Num. Renglón del cónyuge \\
    \hline
    INT0A15 & Personas de 0 a 15 años de edad \\
    \hline
    INT16A64 & Personas de 16 a 64 años de edad \\
    \hline
    INT65A98 & Personas de 65 a 98 años de edad \\
    \hline
    MUJ15A49 & Mujeres de 15 a 49 años de edad \\
    \hline
    MUJER & Mujer \\
    \hline
    HOMBRE & Hombre \\
    \hline
    ACTA\_NAC & Tiene acta de nacimiento \\
    \hline
    TRAB\_SUBOR & En su trabajo principal del mes pasado tuvo un jefe o supervisor \\
    \hline
    TRAB\_INDEP & En su trabajo principal del mes pasado se dedicó a un negocio o actividad por su cuenta \\
    \hline
    TRAB\_PRESTA\_A & Prestación laboral: incapacidad (enfermedad, accidente, maternidad)\\
    \hline
    TRAB\_PRESTA\_B & Prestación laboral: SAR o AFORE \\
    \hline
    TRAB\_PRESTA\_C & Prestación laboral: crédito para vivienda \\
    \hline
    TRAB\_PRESTA\_D & Prestación laboral: guardería \\
    \hline
    TRAB\_PRESTA\_E & Prestación laboral: aguinaldo \\
    \hline
    TRAB\_PRESTA\_F & Prestación laboral: seguro de vida \\
    \hline
    TRAB\_PRESTA\_G & Prestación laboral: no tiene ninguna \\
    \hline
    TRAB\_PRESTA\_H & Prestación laboral: no sabe o no responde \\
    \hline
    C\_SALUD\_HOGA & Donde se atienden los integrantes del hogar A \\
    \hline
    C\_SALUD\_HOGB & Donde se atienden los integrantes del hogar B \\
    \hline
    ENF\_ART & Enfermedad diagnosticada: artritis \\
    \hline
    ENF\_CAN & Enfermedad diagnosticada: cáncer \\
    \hline
    ENF\_CIR & Enfermedad diagnosticada: cirrosis \\
    \hline
    ENF\_REN & Enfermedad diagnosticada: deficiencia renal \\
    \hline
    ENF\_DIA & Enfermedad diagnosticada: diabetes \\
    \hline
    ENF\_COR & Enfermedad diagnosticada: enfermedades del corazón \\
    \hline
    ENF\_PUL & Enfermedad diagnosticada: efisema pulmonar \\
    \hline
    ENF\_VIH & Enfermedad diagnosticada: VIH \\
    \hline
    ENF\_DEF & Enfermedad diagnosticada: deficiencia nutricional (anemia/desnutrición) \\
    \hline
    ENF\_HIP & Enfermedad diagnosticada: hipertensión \\
    \hline
    ENF\_OBE & Enfermedad diagnosticada: obesidad \\
    \hline
    TIENE\_DISCA & Discapacidad tipo: caminar, moverse, subir o bajar escaleras \\
    \hline
    DISCA\_ORI & Discapacidad origen: caminar, moverse, subir o bajar escaleras  \\
    \hline
    DISCA\_GRA & Discapacidad grado:  caminar, moverse, subir o bajar escaleras \\
    \hline
    TIENE\_DISCB & Discapacidad tipo: ver, o solo ve sombras aún usando lentes \\
    \hline
    DISCB\_ORI & Discapacidad origen: ver, o solo ve sombras aún usando lentes \\
    \hline
    DISCB\_GRA & Discapacidad grado: ver, o solo ve sombras aún usando lentes \\
    \hline
    TIENE\_DISCC & Discapacidad tipo: hablar, comunicarse o conversar \\
    \hline
    DISCC\_ORI & Discapacidad origen: hablar, comunicarse o conversar \\
    \hline
    DISCC\_GRA & Discapacidad grado: hablar, comunicarse o conversar \\
    \hline
    TIENE\_DISCD & Discapacidad tipo: oír, aún usando aparato auditivo \\
    \hline
    DISCD\_ORI & Discapacidad origen: oir, aún usando aparato auditivo\\
    \hline
    DISCD\_GRA & Discapacidad grado: oir, aún usando aparato auditivo\\
    \hline
    TIENE\_DISCE & Discapacidad tipo: vestirse, bañarse o comer, desplazarse u otras de cuidado personal\\
    \hline
    DISCE\_ORI & Discapacidad origen: vestirse, bañarse o comer, desplazarse u otras de cuidado personal\\
    \hline
    DISCE\_GRA & Discapacidad grado: vestirse, bañarse o comer, desplazarse u otras de cuidado personal\\
    \hline
    TIENE\_DISCF & Discapacidad tipo: poner atención, aprender cosas, o concentrarse\\
    \hline
    DISCF\_ORI & Discapacidad origen: poner atención, aprender cosas, o concentrarse\\
    \hline
    DISCF\_GRA & Discapacidad grado: poner atención, aprender cosas, o concentrarse\\
    \hline
    OTR\_ING\_A & Otros ingresos: es maestro de escuela de gobierno (federal, estatal, municipal) \\
    \hline
    OTR\_ING\_B & Otros ingresos: es dueño de una tienda\\
    \hline
    OTR\_ING\_C & Otros ingresos: es dueño de algún negocio\\
    \hline
    OTR\_ING\_D & Otros ingresos: es arrendatiario de algún transporte\\
    \hline
    OTR\_ING\_E & Otros ingresos: es doctor(a) o enfermero(a) de gobierno (federal, estatal, municipal)\\
    \hline
    OTR\_ING\_F & Otros ingresos: es servidor público de gobierno (federal, estatal, municipal)\\
    \hline
    OTR\_ING\_G & Otros ingresos: ninguna de las enteriores\\
    \hline
    INAPAM & Tiene tarjeta del instituto nacional de las personas adultas mayores (INAPAM)\\
    \hline
    AM\_A & Recibe dinero por: programa pensión para adultos mayores \\
    \hline
    AM\_B & Recibe dinero por: componente de apoyo para adultos mayores del Programa Prospera\\
    \hline
    AM\_C & Recibe dinero por: otros programas para adultos mayores (estatal o municipal)\\
    \hline
    AM\_D & Recibe dinero por: ningún programa para adultos mayores\\
    \hline
    AM\_E & Recibe dinero por: no sabe/no responde\\
    \hline
    \hline
    \multicolumn{2}{l}{Información socioeconómica de hogares}\\
    \hline
    CON\_REMESA & El hogar recibe dinero proveniente de otros países \\
    \hline
    C\_ESCUSADO & Tipo de baño o escusado de la vivienda \\
    \hline
    USO\_EXC & El baño o escusado es para uso exclusivo de los habitantes de su hogar \\
    \hline
    FOGON\_CHIM & Aparato que usa para cocinar \\
    \hline
    TS\_VHS\_DVD\_BR & Indicadora de tenencia de VHS, DBD o BLU RAY \\
    \hline
    TS\_REFRI & Indicadora de tenencia de refrigerador \\
    \hline
    TS\_VEHI & Indicadora de tenencia de vehículo (carro, camioneta o camión) \\
    \hline
    TS\_MICRO & Indicadora de tenencia de horno (microondas o eléctrico) \\
    \hline
    TS\_TELEFON & Indicadora de tenencia de teléfono (fijo) \\
    \hline
    TS\_COMPU & Indicadora de tenencia de computadora\\
    \hline
    TS\_EST\_GAS & Indicadora de tenencia de estufa / parrilla de gas\\
    \hline
    TS\_INTERNET & Indicadora de tenencia de internet\\
    \hline
    TS\_CELULAR & Indicadora de tenencia de celular\\
    \hline
    TS\_TELEVISION & Indicadora de tenencia de televisión (microondas o eléctrico) \\
    \hline
    C\_SIT\_VIV & Relación con la vivienda en que habita (propia, hipotecada, rentada, etc. ) \\
    \hline
    ESCRITURA1 & Algún integrante del hogar tiene a su nombre las escrituras \\
    \hline
    ESCRITURA2 & Otro integrante del hogar tiene a su nombre las escrituras \\
    \hline
    TIE\_AGRI & Alguna persona del hogar posee o utilizó en los últimos 12 meses tierras para la agricultura o aprovechamiento forestal \\
    \hline
    PROP\_TIERRA1 & Las tierras pertenecen a algún integrante del hogar (propias) \\
    \hline
    PROP\_TIERRA2 & Las tierras pertenecen a más de un integrante del hogar (propias) \\
    \hline
    C\_MAIZ & Tierras: cultiva maiz \\
    \hline
    C\_FRIJ & Tierras: cultiva frijol \\
    \hline
    C\_CERE & Tierras: cultiva cereales \\
    \hline
    C\_FRUT & Tierras: cultiva frutos \\
    \hline
    C\_CANA & Tierras: cultiva caña de azúcar \\
    \hline
    C\_JITO & Tierras: cultiva jitomate \\
    \hline
    C\_CHIL & Tierras: cultiva chile \\
    \hline
    C\_LIMN & Tierras: cultiva limón \\
    \hline
    C\_PAPA & Tierras: cultiva papa \\
    \hline
    C\_CAFE & Tierras: cultiva café \\
    \hline
    C\_CATE & Tierras: cultiva aguacate \\
    \hline
    C\_FORR & Tierras: cultiva forrajes \\
    \hline
    C\_NING & Tierras: no se cultiva \\
    \hline
    CUL\_RIEGO & Para cultivar utiliza: sistemas de riego \\
    \hline
    CUL\_MAQUINA & Para cultivar utiliza: maquinaria (tractor y/o otros)\\
    \hline
    CUL\_ANIM & Para cultivar utiliza: ayuda de animales\\
    \hline
    CUL\_FERORG & Para cultivar utiliza: composta / fertilizantes orgánicos \\
    \hline
    CUL\_FERQUIM & Para cultivar utiliza: fertilizantes químicos \\
    \hline
    CUL\_PLAGUI & Para cultivar utiliza: plaguicidas\\
    \hline
    CUL\_RIEGO & Para cultivar utiliza: sistemas de riego \\
    \hline
    USO\_HID\_TRA & En el hogar se emplea la hidroponía o la agricultura de traspatio (huertos) para el cultivo de productos \\
    \hline
    PROYECTO & Algún integrante del hogar le gustaría realizar un proyecto productivo o de servicio \\
    \hline
    P\_AGRI & Proyecto productivo: agricultura, cría y explotación de animales, aprovechamiento forestal, pesca y caza \\
    \hline
    P\_MANU & Proyecto productivo: manufactura (elaboración de productos) \\
    \hline
    P\_COME & Proyecto productivo: comercio (compra-venta de bienes)\\
    \hline
    P\_TRAN & Proyecto productivo: transporte (mercancías o personas)\\
    \hline
    P\_PROF & Proyecto productivo: servicios profesionales, científicos y/o técnicos (oficios)\\
    \hline
    P\_EDUC & Proyecto productivo: servicios educativos (capacitación) \\
    \hline
    P\_SALD & Proyecto productivo: servicios de salud y de asistencia social (enfermería, cuidado de personas)\\
    \hline
    P\_RECR & Proyecto productivo: servicios de esparcimiento, culturales y deportivos, y otros servicios recreativos\\
    \hline
    P\_ALOJ & Proyecto productivo: servicios de alojamiento temporal y de preparación de alimentos y bebidas\\
    \hline
    P\_COMU & Proyecto productivo: servicios de telecomunicaciones (café internet, casetas telefónicas)\\
    \hline
    P\_OTRO & Proyecto productivo: otro\\
    \hline
    TIPO\_PROY\_ESP &  Especificar otro proyecto \\
    \hline
    EJERCICIO & Año de captura de la encuesta \\
    \hline
    CONSISTENCIA & Índice de consistencia de la información \\
    \hline
    VIGENCIA & Índice de la vigencia de la información \\
    \hline
    USABILIDAD & Índice de la usabilidad de la información \\
    \hline
    INT\_CARE & Índice de la intensidad de la carencia \\
    \hline
\end{longtable}
\captionof{table}{Catálogo de variables socioeconómicas SIFODE 39.9}
\label{tab:varsocioe}

\subsection{SIFODE Domicilio}
Otra sección importante dentro de la información recolectada por el SIFODE es la correspondiente a la geolocalización de los hogares, la cual se encuentra en forma de domicilio y algunas referencias.

\begin{longtable}{|p{8cm}|p{8cm}|}
    \hline
    \hline
    Campo  & Descripción\\
    \hline
    \hline
    \multicolumn{2}{l}{Identificadores únicos}\\
    \hline
    LLAVE\_HOGAR\_H & Identifica encuestas únicas integradas por el o los folios que identifican a una encuesta de manera única declaradas por el programa\\
    \hline
    C\_INTEGRANTE & Número de renglón de integrante en la encuesta\\
    \hline
    ID\_UNICO\_39\_9 & Identificador Único de Personas en Padrones corresponiente a SIFODE 39.9\\
    \hline
    \hline
    \multicolumn{2}{l}{Ubicación de personas y hogares}\\
    \hline
    S\_CVE\_ENTIDAD\_FEDERATIVA & Clave de entidad \\
    \hline
    S\_NOM\_ENTIDAD\_FEDERATIVA & Entidad federativa \\
    \hline
    S\_CVE\_MUNICIPIO & Clave de municipio\\
    \hline
    S\_NOM\_MUNICIPIO & Municipio o delegación\\
    \hline
    S\_CVE\_LOCALIDAD & Clave de localidad\\
    \hline
    S\_NOM\_LOCALIDAD & Localidad \\
    \hline
    S\_TIPOLOC & Tipo de localidad \\
    \hline
    S\_LONGITUD & Longitud \\
    \hline
    S\_LATITUD & Latitud \\
    \hline
    S\_CVE\_AGEB & Clave de la cartografía censal al cierre del censo de población y vivienda 2010 \\
    \hline
    S\_CVE\_MANZANA & Clave Manzana \\
    \hline
    S\_C\_TIPO\_VIAL & Tipo de vialidad \\
    \hline
    S\_NOMBRE\_VIAL & Nombre de vialidad \\
    \hline
    S\_CP & Código postal \\
    \hline
    S\_C\_TIPO\_ASENTAMIENTO & Tipo de asentamiento humano \\
    \hline
    S\_NOMBRE\_ENTRE\_VIAL\_1 & Nombre de vialidad entre la calle 1 \\
    \hline
    S\_C\_TIPO\_ENTRE\_VIAL\_1 & Tipo de vialidad 1 \\
    \hline
    S\_C\_TIPO\_VIAL\_POS & Nombre de vialidad entre la calle 2 \\
    \hline
    S\_NOMBRE\_ENTRE\_VIAL\_2 & Nombre de vialidad entre la calle 2 \\
    \hline
    S\_C\_TIPO\_ENTRE\_VIAL\_2 & Codificación tipo de vialidad \\
    \hline
    S\_NOMBRE\_VIAL\_POS & Nombre de la vialidad la calle de atrás \\
    \hline
    S\_DESC\_UBIC & Descripción de la ubicación \\
    \hline
    S\_NUM\_EXT & Número exterior \\
    \hline
    S\_LETRA\_EXT & Letra exterior \\
    \hline
    NUMEXTNUM2 & Número exterior anterior \\
    \hline
    S\_NUM\_INT & Número interior \\
    \hline
    S\_LETRA\_INT & Letra interior \\
    \hline
    S\_ORIGEN\_DOMICILIO & Origen del domicilio \\
    \hline
    S\_CARRETERA & Nombre compuesto de la carretera, conforme a la norma técinca sobre domicilios geográficos del INEGI \\
    \hline
    S\_CAMINO & Nombre compuesto del Camino, conforme a la Norma técnica sobre domicilios geográficos del INEGI \\
    \hline
\end{longtable}
\captionof{table}{Información domiciliaria de SIFODE 39.9}
\label{tab:domicilios}
